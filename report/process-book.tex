\documentclass[a4paper,10pt]{article}
\usepackage[utf8]{inputenc}
\usepackage{fullpage}
\usepackage[T1]{fontenc}
\usepackage{graphicx}

\begin{document}
\title{Process book: DataMoviz}
\author{Quentin de Longraye, Aymen Gannouni, Victor Le}

\maketitle

\tableofcontents

\newpage

\section{Presentation of DataMoviz}

\subsection{Overview, motivation and target audience}

Overview, motivation, target audience
+ Questions: What am I trying to show this my viz?

\subsection{Related work and inspiration}

Related work and inspiration

\section{Exploratory analysis}

\subsection{Dataset}

Presentation of the dataset.
Dataset: where does it come from, what are you processing steps?

\subsection{Data previsualization}

Exploratory data analysis: What viz have you used to gain insights on the data?

\section{Solution's build-up}

\subsection{Considered visualizations}

Designs: What are the different visualizations you considered? Justify the design decisions you made using the perceptual and design principles.
Did you deviate from your initial proposal?

\subsection{Implementation}

Implementation: Describe the intent and functionality of the interactive visualizations you implemented. Provide clear and well-referenced images showing the key design and interaction elements.

\subsection{Evaluation}

Evaluation: What did you learn about the data by using your visualizations? How did you answer your questions? How well does your visualization work, and how could you further improve it?

\section{Peer assessment}

\begin{itemize}
  \item Preparation – were they prepared during team meetings?
  \item Contribution – did they contribute productively to the team discussion and work?
  \item Respect for others’ ideas – did they encourage others to contribute their ideas?
  \item Flexibility – were they flexible when disagreements occurred?
\end{itemize}

\end{document}
